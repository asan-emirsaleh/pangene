\documentclass{bioinfo}
\copyrightyear{2024}
\pubyear{2024}

\usepackage{graphicx}
\usepackage{hyperref}
\usepackage{url}
\usepackage{tabularx}
\usepackage{amsmath}
\usepackage[ruled,vlined]{algorithm2e}
\newcommand\mycommfont[1]{\footnotesize\rmfamily{\it #1}}
\SetCommentSty{mycommfont}
\SetKwComment{Comment}{$\triangleright$\ }{}
\renewcommand{\ttdefault}{cmtt}

\usepackage{natbib}
\bibliographystyle{apalike}

\DeclareMathOperator*{\argmax}{argmax}

\begin{document}
\firstpage{1}

\title[Pangene graphs]{Studying gene content with pangenome gene graphs}
\author[Li]{Heng Li$^{1,2}$}
\address{$^1$Dana-Farber Cancer Institute, 450 Brookline Ave, Boston, MA 02215, USA,
$^2$Harvard Medical School, 10 Shattuck St, Boston, MA 02215, USA}

\maketitle

\begin{abstract}

\section{Availability and implementation:}
\href{https://github.com/lh3/pangene}{https://github.com/lh3/pangene}

\section{Contact:} hli@ds.dfci.harvard.edu
\end{abstract}

\section*{Introduction}

\begin{methods}
\section*{Methods}
\end{methods}

\section*{Results}

\subsection*{Finding structural variants betwen two human genomes}

We downloaded GenCode comprehensive human annotation v44,
retained the protein coding genes and filtered out readthrough transcripts and mitochondrial genes.
Because GenCode does not include HLA-DRB3, HLA-DRB4 and several KIR genes,
we manually added 20 HLA-DRB genes from the IPD-IMGT/HLA database and
15 KIR genes from the IPD-KIR database.
In the end, we constructed a protein set with 109,317 proteins, representing 19,335 protein coding genes.

We aligned the proteins to the human reference genome GRCh38~\citep{Schneider:2017aa} and T2T-CHM13~\citep{Nurk:2022up}
with miniprot~\citep{Li:2023ac} v0.12 under option ``{\tt --outs=0.97 -Iu}''
and constructed a pangene graph.
The resulting graph contains 19,058 genes and 19,361 edges.
We identified 104 large-scale variants, represented by ``bubbles'' in the graph,
involving 1,026 protein coding genes affected by gene copy, gene order or gene orientation changes between the two genome.
These genes include many known events such as PDPR, SMN2, CTAGE9, HPR, ORM1, CCL4, NCF1 and Amylase~\citep{Handsaker:2015ur,Sudmant:2010aa}
as well as new events.
We manually checked the miniprot alignment of 20 relatively small events
and believed pangene is reporting the desired haplotype structures.
Some of the large gene clusters affected by structural changes, such as SMN2 and Amylase,
would not be easily captured by whole-genome alignment as they are not represented by colinear alignments.

\subsection*{The pangene of \emph{M. tuberculosis} strains}

We downloaded the \emph{M. tuberculosis} reference strain H37Rv and its gene annotation from RefSeq (AC:GCF\_000195955.2),
which included 3,906 protein sequences.
We obtained the complete long-read assemblies of 145 other strains from \citet{Marin:2022aa}.
Following the instruction of Panaroo~\citep{Tonkin-Hill:2020aa},
we ran Prodigal~\citep{Hyatt:2010aa} v2.6.3 on the reference strain to train the Prodigal model
and ran Prokka~\citep{Seemann:2014aa} v1.14.6 with the pretrained model to predict protein coding genes
in the 145 non-reference strains.
We used CD-HIT~\citep{Li:2006aa,Fu:2012aa} v4.8.1 with option ``{\tt -c 0.98}'' to cluster non-reference protein sequences,
which resulted in 6,875 clusters.
We constructed the final protein set by merging RefSeq annotations and the protein clusters
and mapped these proteins to each \emph{M. tuberculosis} genome with miniprot.
We finally ran pangene with option ``{\tt -P H37Rv.id.txt -p.001}'',
where option ``{\tt -P}'' asks pangene to prioritize on genes from H37Rv and
``{\tt -p}'' tries to keep all genes regardless of their frequency in the pangenome.

Pangene constructed a graph consisting of 4,215 genes,
3,690 of which were considered to be present in all 146 genomes.


For comparison, we also ran the strict mode of Panaroo v1.3.4 with option ``{\tt --clean-mode strict}''.


\section*{Discussions}

\section*{Acknowledgements}

{\bf Funding:} NHGRI R01HG010040 and Chan-Zuckerberg Initiative

\bibliography{pangene}

\end{document}
